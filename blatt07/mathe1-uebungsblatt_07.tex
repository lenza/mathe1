\documentclass[11pt,a4paper]{article}

\usepackage{german}
\usepackage{amssymb}
\usepackage{amsmath}
\usepackage{epsfig}
\usepackage{graphics}
\usepackage{moreverb}
\usepackage[utf8]{inputenc}
% -----------------------------------------------------------------------
%
% Page layout
%

% gieza bigger page
\setlength{\textwidth}{16cm}
\setlength{\topmargin}{-1cm}
\setlength{\evensidemargin}{0cm}
\setlength{\oddsidemargin}{0cm}
\setlength{\textheight}{24cm}

% skip between paragraphs
\setlength{\parskip}{1ex}
% ... and no indentation at start of a new paragraph
\setlength{\parindent}{0ex}

\pagestyle{plain}
\thispagestyle{plain}

% -----------------------------------------------------------------------
%
% Macro f�r das Abgabeblatt
%
\newcommand{\Abgabeblatt}[8]
          {\Arbblatt{#1}{#2}{#3}{#4}{#5}{#6}{#7}{#8}{"Ubungsblatt}}

\newcommand{\grenzop}[3]{\mathop{#1}\limits^{#2}_{#3}}

\newcommand{\langrechtspfeil}[2]{\grenzop{\longrightarrow}{#1}{#2}}
\newcommand{\bob}[0]{\hspace*{\fill}$\Box$}

% -----------------------------------------------------------------------

\newcommand{\Arbblatt}[9]
{
\begin{tabular}[t]{lr}
  \begin{tabular}[t]{l}
    Mathematik I\hspace{25ex} WiSe 2016/2017\\ Tutor: {#3}\\ \hspace{75ex} \\
    {\Huge\sf {#9} {#1} } \\[1.5ex]
    {\Large Aufgabenl"osung} \\[1ex]
    {\Large Abgabe: {#2}} \\[1ex]
  \end{tabular}
  &
  \begin{tabular}[t]{l}
      {#4}\\
      {#5}\\
      {#6}\\
      {#7}\\
      {#8}\\
  \end{tabular} \\[2ex]
\end{tabular}\\
\rule{\linewidth}{1pt}
}

%%% Local Variables: 
%%% mode: latex
%%% TeX-master: t
%%% End: 

\begin{document}

\Abgabeblatt{7}{06. Dezember 2016}{Niklas Gather \#10}
                {Lena Zamzow}{Tobias Pfannschmidt}{Daniel Zaccaria}{}{}


%%%%%%%%%%%%%%%%%%%%%%%%%%%%%%%%%%%%%%%%%%%%%%%%
%%%%%%%%%%%%%%%%%%%%%%%%%%%%%%%%%%%%%%%%%%%%%%%%
\section*{Aufgabe H16}
$x=1981, y=1532$\\
$a,b \in \mathbb{Z}$, s.d. $a\cdot x + b\cdot y = ggT(x,y)$\\
\begin{center}
	\begin{tabular}{ | l | l | l | l | l | l | l|}
		\hline
		i&x&y&q&a&b&$b_i=a_{i+1}-q_i \cdot b_{i+1} $ \\ \hline		
		0&1981&1532&1&737&-953&$-216-(1 \cdot 737 )$ \\ \hline
		1&1532&449&3&-216&737&$89-(3 \cdot (-216) )$ \\ \hline
		2&449&185&2&89&-216&$-38-(2 \cdot 89)$\\ \hline
		3&185&79&2&-38&89&$13-(2 \cdot (-38))$ \\ \hline
		4&79&27&2&13&-38&$-12-(2 \cdot 13)$ \\ \hline
		5&27&25&1&-12&13&$1-(1\cdot (-12))$ \\ \hline
		6&25&2&12&1&-12&$0-(12 \cdot 1)$\\ \hline
		7&2&1&2&0&1&$1-(2 \cdot 0)$\\ \hline
		8&1&0&&1&0&\\ \hline	
	\end{tabular}
\end{center}
Überprüfung:\\
$a\cdot x + b\cdot y = ggT(x,y)$\\
$737\cdot 1981+((-953)\cdot 1532)=1$\\
Der größte gemeinsame Teiler von x und y, nach der Anwendung des erweiterten euklidischen Algorithmus, ist 1.\\

\section*{Aufgabe H17}
Seien für $n>0$ die Fibonacci-Zahlen $F_n$ wie auf Blatt 6 definiert.\\

$F_0 = 1$\\
$F_1 = 1$\\
$F_n=F_{n-1}+F_{n-2}$ s.d.\\
$F_2 = 2$\\
$F_3 = 3$\\
$F_4 = 5$\\
$F_5 = 8$\\
$F_6 = 13$\\

\subsection*{a)}
Der $ggT(F_n, F_{n+1})$ mit $n \in \{3,4,5,6\}.$ Es soll gelten: $ggT(F_n, F_{n+1})=a\cdot F_n +b \cdot F_{n+1}$\\
\begin{center}
	\begin{tabular}{ | l | l | l | l | l | l | l|}
		\hline
		i&$F_3$&$F_4$&q&a&b&$b_i=a_{i+1}-q_i \cdot b_{i+1} $ \\
		\hline \hline
		0&3&5&0&2&-1&$-1-(0 \cdot 2 )$ \\ \hline
		1&5&3&1&-1&2&$1-(1 \cdot (-1))$ \\ \hline
		2&3&2&1&1&-1&$0-(1 \cdot 1)$ \\ \hline
		3&2&1&2&0&1&$1-(2\cdot 0)$ \\ \hline
		4&1&0&&1&0&\\ \hline
	\end{tabular}
\end{center}

Überprüfung:\\
$a\cdot F_3 +b \cdot F_4 = ggT(F_3,F_4)=(2\cdot 3)+((-1)\cdot5)=1$\\
Nach der Anwendung des erweiterten euklidischen Algorithmus ist der größte gemeinsame Teiler der dritten und vierten Fibonacci-Zahlen $(F_3~und~F_4)$ gleich 1.\\

\begin{center}
	\begin{tabular}{ | l | l | l | l | l | l | l|}
		\hline
		i&$F_4$&$F_5$&q&a&b&$b_i=a_{i+1}-q_i \cdot b_{i+1}$ \\ \hline	 \hline	
		0&5&8&0&-3&2&$2-(0 \cdot 1 )$ \\ \hline
		1&8&5&1&2&-3&$-1-(1 \cdot 2)$\\ \hline
		2&5&3&1&-1&2&$1-(1 \cdot (-1))$ \\ \hline
		3&3&2&1&1&-1&$0-(1 \cdot 1)$ \\ \hline
		4&2&1&2&0&1&$1-(2\cdot 0)$ \\ \hline
		5&1&0&&1&0&\\ \hline	
	\end{tabular}
\end{center}

Überprüfung:\\
$a\cdot F_4 +b \cdot F_5 = ggT(F_5,F_6)=((-3)\cdot5)+(2\cdot8)=1$\\
Nach der Anwendung des erweiterten euklidischen Algorithmus ist der größte gemeinsame Teiler der vierten und fünften Fibonacci-Zahlen $(F_4~und~F_5)$ gleich 1.\\

\begin{center}
	\begin{tabular}{ | l | l | l | l | l | l | l|}
		\hline
		i&$F_5$&$F_6$&q&a&b&$b_i=a_{i+1}-q_i \cdot b_{i+1} $ \\ \hline	\hline	
		0&8&13&0&5&-3&$-3-(0 \cdot 5 )$ \\ \hline
		1&13&8&1&-3&5&$2-(1 \cdot (-3) )$ \\ \hline
		2&8&5&1&2&-3&$-1-(1 \cdot 2)$\\ \hline
		3&5&3&1&-1&2&$1-(1 \cdot (-1))$ \\ \hline
		4&3&2&1&1&-1&$0-(1 \cdot 1)$ \\ \hline
		5&2&1&2&0&1&$1-(2\cdot 0)$ \\ \hline
		6&1&0&&1&0&\\ \hline
	\end{tabular}
\end{center}

Überprüfung:\\
$a\cdot F_5 +b \cdot F_6 = ggT(F_5,F_6) = 5 \cdot 8 + ((-3)\cdot 13) = 40+(-39)=1$\\
Nach der Anwendung des erweiterten euklidischen Algorithmus ist der größte gemeinsame Teiler der fünften und sechsten Fibonacci-Zahlen $(F_5~und~F_6)$ gleich 1.\\

\section*{Aufgabe H17 b)}
\textbf{Behauptung}: \\
Der ggT$(F_n, F_{n+1})$ ist immer 1.\\
Für $\forall~n \geq 3$ ist $a=F_{n-1} , b= F_{n-2}$\\
\\
Beweis durch vollständige Induktion über n.\\
\textbf{Induktionsanfang}:\\
$n<3$
\\
\textbf{Induktionsvoraussetzung}:\\
$ggT(F_n, F_{n+1}) = F_{n-1} \cdot F_n+F_{n-2}\cdot F_{n+1}$\\
\\
\textbf{Induktionsschritt}:\\
z.Z.
$ggT(F_{n+1}, F_{(n+1)+1})=F_{(n+1)-1} \cdot F_{n+1} + F_{(n+1)-2} \cdot F_{(n+1)+1}$\\
$=ggT(F_{n+1}, F_{n+2})=F_{n} \cdot F_{n+1} + F_{n-1} \cdot F_{n+2}$\\
$=ggT(F_{n+1}, F_{n+2})=(F_{n} \cdot F_{n} + F_n \cdot F_{n-1}) + (F_{n-1} \cdot F_{n+1} + F_{n-1} \cdot F_{n}$)\\
$=ggT(F_{n+1}, F_{n+2})=F_{n} (F_{n} + F_{n-1}) + F_{n-1} (F_{n+1} + F_{n}$) (Distributivgesetz nach 4.22)\\

\end{document}
