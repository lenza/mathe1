\documentclass[11pt,a4paper]{article}

\usepackage{german}
\usepackage{amssymb}
\usepackage{amsmath}
\usepackage{epsfig}
\usepackage{graphics}
\usepackage{moreverb}
\usepackage[utf8]{inputenc}
% -----------------------------------------------------------------------
%
% Page layout
%

% gieza bigger page
\setlength{\textwidth}{16cm}
\setlength{\topmargin}{-1cm}
\setlength{\evensidemargin}{0cm}
\setlength{\oddsidemargin}{0cm}
\setlength{\textheight}{24cm}

% skip between paragraphs
\setlength{\parskip}{1ex}
% ... and no indentation at start of a new paragraph
\setlength{\parindent}{0ex}

\pagestyle{plain}
\thispagestyle{plain}

% -----------------------------------------------------------------------
%
% Macro f�r das Abgabeblatt
%
\newcommand{\Abgabeblatt}[8]
          {\Arbblatt{#1}{#2}{#3}{#4}{#5}{#6}{#7}{#8}{"Ubungsblatt}}

\newcommand{\grenzop}[3]{\mathop{#1}\limits^{#2}_{#3}}

\newcommand{\langrechtspfeil}[2]{\grenzop{\longrightarrow}{#1}{#2}}
\newcommand{\bob}[0]{\hspace*{\fill}$\Box$}

% -----------------------------------------------------------------------

\newcommand{\Arbblatt}[9]
{
\begin{tabular}[t]{lr}
  \begin{tabular}[t]{l}
    Mathematik I\hspace{25ex} WiSe 2016/2017\\ Tutor: {#3}\\ \hspace{75ex} \\
    {\Huge\sf {#9} {#1} } \\[1.5ex]
    {\Large Aufgabenl"osung} \\[1ex]
    {\Large Abgabe: {#2}} \\[1ex]
  \end{tabular}
  &
  \begin{tabular}[t]{l}
      {#4}\\
      {#5}\\
      {#6}\\
      {#7}\\
      {#8}\\
  \end{tabular} \\[2ex]
\end{tabular}\\
\rule{\linewidth}{1pt}
}

%%% Local Variables: 
%%% mode: latex
%%% TeX-master: t
%%% End: 

\begin{document}

\Abgabeblatt{8}{13. Dezember 2016}{Niklas Gather \#10}
                {Lena Zamzow}{Tobias Pfannschmidt}{}{}{}


%%%%%%%%%%%%%%%%%%%%%%%%%%%%%%%%%%%%%%%%%%%%%%%%
%%%%%%%%%%%%%%%%%%%%%%%%%%%%%%%%%%%%%%%%%%%%%%%%
\section*{Aufgabe H18}
$x\equiv1(mod 3)$\\
$x\equiv3(mod 5)$\\
$x\equiv5 (mod 11)$\\
$\overline{m_1}=m_2\cdot m_3$   $\overline{m_1} = 5 \cdot 11 = 55$\\
$\overline{m_2}=m_1\cdot m_3$   $\overline{m_2 = 3} \cdot 11 = 33$\\
$\overline{m_3}=m_2\cdot m_1$   $\overline{m_3 = 3} \cdot 5 = 15$\\
\\
Suchen $x_1, x_2, x_3 \in \mathbb{Z}$ s.d.\\
$\overline{m_1}\cdot x_1 + \overline{x_2} \cdot x_2 + \overline{m_3} \cdot x_3 = 1(mod(m_1,m_2,m_3))$\\
\\
Suchen $x_1,y_1 aus \mathbb{Z}:$\\

\begin{center}
	\begin{tabular}{ | l | l | l | l | l | l | l|}
		\hline
		i&$\overline{m_1}$&$m_1$&q&x&y&$b_i=x_{i+1}-q_i \cdot y_{i+1} $ \\ \hline \hline		
		0&55&3&18&1&-18&$0-(1 \cdot 18 )$ \\ \hline
		1&3&1&3&0&1&$1-(3 \cdot 0 )$ \\ \hline
		2&1&0&&1&0&\\ \hline	
	\end{tabular}
\end{center}
$\overline{m_1} \cdot x_1 + m_1 \cdot y_1 = 1$\\
$55 ßcdot 1 + 3\cdot (-18) = 1 \checkmark$
\\
Suchen $x_2,y_2 aus \mathbb{Z}:$\\
\begin{center}
	\begin{tabular}{ | l | l | l | l | l | l | l|}
		\hline
		i&$\overline{m_2}$&$m_2$&q&x&y&$b_i=x_{i+1}-q_i \cdot y_{i+1} $ \\ \hline \hline
		0&33&5&6&2&-13&$-1-(2 \cdot 6 )$ \\ \hline
		1&5&3&1&-1&2&$1-(1 \cdot (-1))$ \\ \hline
		2&3&2&1&1&-1&$0-(1 \cdot 1)$ \\ \hline
		3&2&1&2&0&1&$1-(2\cdot 0)$ \\ \hline
		4&1&0&&1&0&\\ \hline
	\end{tabular}
\end{center}
$\overline{m_2} \cdot x_2 + m_2 \cdot y_2 = 1$\\
$33\cdot 2 + 5 \cdot (-13) = 66-65 = 1 \checkmark$\\
\\
Suchen $x_3,y_3 aus \mathbb{Z}:$\\
\begin{center}
	\begin{tabular}{ | l | l | l | l | l | l | l|}
		\hline
		i&$\overline{m_3}$&$m_3$&q&x&y&$b_i=x_{i+1}-q_i \cdot y_{i+1} $ \\ \hline \hline
		0&15&11&1&3&-4&$-1-(1 \cdot 3 )$ \\ \hline
		1&11&4&2&-1&3&$1-(2 \cdot (-1))$ \\ \hline
		2&4&3&1&1&-1&$0-(1 \cdot 1)$ \\ \hline
		3&3&1&3&0&1&$1-(3\cdot 0)$ \\ \hline
		4&1&0&&1&0&\\ \hline
	\end{tabular}
\end{center}

$\overline{m_3} \cdot x_3 + m_3 \cdot y_3 = 1$\\
$15 \cdot 3 + 11 \cdot (-4) = 45 - 44 = 1 \checkmark$\\
\\
$\overline{m_1}\cdot x_1 + \overline{x_2} \cdot x_2 + \overline{m_3} \cdot x_3 = 1(mod(m_1,m_2,m_3))$\\
$55 \cdot 1 + 33 \cdot 2 + 15 \cdot 3 = 166 \checkmark$\\
$166\equiv 1 (mod(3,5,11)) = 166 \equiv 1(mod(165))$\\
\\
$t=b_1 \cdot \overline{m_1} \cdot x_1 + b_2 \cdot \overline{m_2} \cdot x_2 + b_3 \cdot \overline{m_3} \cdot x_3$\\
$= 1 \cdot 55 \cdot 1 + 3 \cdot 33 \cdot 2 + 5 \cdot 15 \cdot 3$\\
$= 55 + 198 + 225$\\
$= 478(mod(3,5,11))$\\
t $ mod(3,5,11) = 148 $
Eindeutig, weil 148 ist immer der Rest ist.

\section*{Aufgabe H19}
\section*{Aufgabe H20}
\section*{Aufgabe B4}

\end{document}
