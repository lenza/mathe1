\documentclass[11pt,a4paper]{article}

\usepackage{german}
\usepackage{amssymb}
\usepackage{amsmath}
\usepackage{epsfig}
\usepackage{graphics}
\usepackage{moreverb}
\usepackage[utf8]{inputenc}
% -----------------------------------------------------------------------
%
% Page layout
%

% gieza bigger page
\setlength{\textwidth}{16cm}
\setlength{\topmargin}{-1cm}
\setlength{\evensidemargin}{0cm}
\setlength{\oddsidemargin}{0cm}
\setlength{\textheight}{24cm}

% skip between paragraphs
\setlength{\parskip}{1ex}
% ... and no indentation at start of a new paragraph
\setlength{\parindent}{0ex}

\pagestyle{plain}
\thispagestyle{plain}

% -----------------------------------------------------------------------
%
% Macro f�r das Abgabeblatt
%
\newcommand{\Abgabeblatt}[8]
          {\Arbblatt{#1}{#2}{#3}{#4}{#5}{#6}{#7}{#8}{"Ubungsblatt}}

\newcommand{\grenzop}[3]{\mathop{#1}\limits^{#2}_{#3}}

\newcommand{\langrechtspfeil}[2]{\grenzop{\longrightarrow}{#1}{#2}}
\newcommand{\bob}[0]{\hspace*{\fill}$\Box$}

% -----------------------------------------------------------------------

\newcommand{\Arbblatt}[9]
{
\begin{tabular}[t]{lr}
  \begin{tabular}[t]{l}
    Mathematik I\hspace{25ex} WiSe 2016/2017\\ Tutor: {#3}\\ \hspace{75ex} \\
    {\Huge\sf {#9} {#1} } \\[1.5ex]
    {\Large Aufgabenl"osung} \\[1ex]
    {\Large Abgabe: {#2}} \\[1ex]
  \end{tabular}
  &
  \begin{tabular}[t]{l}
      {#4}\\
      {#5}\\
      {#6}\\
      {#7}\\
      {#8}\\
  \end{tabular} \\[2ex]
\end{tabular}\\
\rule{\linewidth}{1pt}
}

%%% Local Variables: 
%%% mode: latex
%%% TeX-master: t
%%% End: 

\begin{document}

\Abgabeblatt{5}{22. November 2016}{Niklas Gather \#10}
                {Lena Zamzow}{Tobias Pfannschmidt}{Daniel Zaccaria}{}{}


%%%%%%%%%%%%%%%%%%%%%%%%%%%%%%%%%%%%%%%%%%%%%%%%
%%%%%%%%%%%%%%%%%%%%%%%%%%%%%%%%%%%%%%%%%%%%%%%%

\section*{Aufgabe H11}
$f:X \longrightarrow Y$ $ g:Y \longrightarrow Z$\\
\\
f sei surjektiv, d.h. für jedes y $\in $ Y gibt es ein x $\in$ X : \\
$x \longrightarrow y$ s.d. $f(x)=y$\\
\\
$g\circ f$ sei injektiv, d.h es gibt genau ein Abbild von z $ \in X: $\\
$ X \longrightarrow Z~ $\\
$\forall x \in X gilt~g(f(x1))=g(f(x2)) \Rightarrow x1 = x2$\\ 

$g(f(x1))=g(f(x2)) \Rightarrow x1=x2 \wedge \exists x \in x ~ s.d. ~f(x)=y$\\ 
$ \Rightarrow g(f(x1)) = g(f(x2))\Rightarrow f(x1)=f(x2)$\\
$\Rightarrow$ g injektiv   $_\Box$

\section*{Aufgabe H12}

Wir wählen für X die Menge aller natürlichen Zahlen $\mathbb{N}$. Dann gilt, dass es zu jedem Funktionswert f(x) genau ein Urbild in der Menge X gibt und f somit injektiv ist.
\\ Wir können für die Menge X jede beliebige Teilmenge wählen von: $\{x \in \mathbb{R} ~| ~ x \geq 0\}$ (also auch jede beliebige Teilmenge von $\mathbb{N}$).
\\Wir können dagegen nicht die Menge der ganzen Zahlen $\mathbb{Z}$ wählen, da diese Menge auch negative Zahlen enthält.
\\ Zum Beispiel ergibt sich daraus: $(-2)^2 - 1 = 3 = 2^2 -1 \Rightarrow$ f nicht injektiv, da f(x)=3 von zwei verschiedenen Elementen in X abgebildet wird.

\section*{Aufgabe H13}

\subsection*{a)}
Für alle n $ \in \mathbb{N}$ gilt: $6 |  n^3-n $ \\
Beweis durch Induktion über n:\\
\subsubsection*{Induktionsanfang}
$n=0$\\
\ \\
$0^3-0 = 0 ~ \surd$
\\ (Nach Bsp. 4.3 VL gilt $\forall m\in \mathbb{Z}: m|0$)
\subsubsection*{Induktionsvoraussetzung}
Zu zeigen: Für ein n $\in N$ gilt $n^3-n$ ist teilbar durch 6\\
\subsubsection*{Induktionsschritt}
Induktion für \(n+1\)
\begin{align*}
(n+1)^3 - (n+1) \\
= & ~ (n+1)^2(n+1) - (n+1) \\
= & ~ (n^2 + 2n + 1)(n+1) - (n + 1) \\
= & ~ (n^2 + 2n + 1)(n+1) - n - 1  \text{~~~(binomische Formel)} \\
= & ~ n^3 + 2n^2 + n + n^2 + 2n + 1  -n -1 \\
= & ~ (n^3 - n) + 3n^2 + 3n\\
= ~ (n^3 - n) + 3(n^2 + n) \\
= & ~ \underbrace{(n^3 - n)}_\text{teilbar durch 6 laut Induktionsvoraussetzung} + \underbrace{3~ \overbrace{(n^2 + n)}}_\text{teilbar durch 3 wg. Faktor}^\text{teilbar durch zwei (s.u.)}
\end{align*}

Der erste Teil des Termes der letzten Zeile ist nach Induktionsvoraussetzung durch 6 teilbar. Da der Teilterm \(3(n^2 + n)\) des Gesamtterms zum einen durch \(3\) teilbar ist (wegen des Faktors 3) und zum zweiten \((n^2+n)\) immer durch 2 teilbar ist (das Quadrat einer natürlichen Zahl addiert mit sich selbst ergibt immer eine gerade Zahl), ist der gesamte zweite Teilterm ebenfalls durch 6 teilbar (da \( 2 \cdot 3 = 6\)). Die Summe zweier Terme, die durch 6 teilbar sind, ist ebenfalls durch 6 teilbar. $_\Box$

\subsection*{b)}

Behauptung: $\forall n \in \mathbb{N}$ gilt: $\sum \limits_{i=0}^n 2^i = 2^{n+1}-1$
\\ Beweis durch Vollständige Induktion über n.

\textbf{Induktionsanfang}:
\\n=0
\\ $\sum \limits_{i=0}^0 2^0 = 0 = 2^{0+1}-1$ $\surd$

\textbf{Induktionsvoraussetzung}:
\\ Für beliebiges $n \in \mathbb{N}$ gelte:  $\sum \limits_{i=0}^n 2^i = 2^{n+1}-1$

\newpage
\textbf{Induktionsschritt}:
\\zu zeigen (unter der Induktionsvoraussetzung): $\sum \limits_{i=0}^{n+1} 2^i = 2^{(n+1)+1}-1$

$\sum \limits_{i=0}^{n+1} 2^i$
\\ $= 2^{n+1} + \sum \limits_{i=0}^{n} 2^i  $
\\ $= 2^{n+1} + (2^{n+1}-1) ~~ (nach ~ \textbf{Induktionsvoraussetzung})$
\\ $= 2 \times 2^{n+1}-1$
\\ $= 2^{n+2}-1$
\\ $= 2^{(n+1)+1}-1 ~~ _\Box$

\end{document}
