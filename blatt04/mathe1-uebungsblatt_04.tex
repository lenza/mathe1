\documentclass[11pt,a4paper]{article}

\usepackage{german}
\usepackage{amssymb}
\usepackage{amsmath}
\usepackage{epsfig}
\usepackage{graphics}
\usepackage{moreverb}
\usepackage[utf8]{inputenc}
% -----------------------------------------------------------------------
%
% Page layout
%

% gieza bigger page
\setlength{\textwidth}{16cm}
\setlength{\topmargin}{-1cm}
\setlength{\evensidemargin}{0cm}
\setlength{\oddsidemargin}{0cm}
\setlength{\textheight}{24cm}

% skip between paragraphs
\setlength{\parskip}{1ex}
% ... and no indentation at start of a new paragraph
\setlength{\parindent}{0ex}

\pagestyle{plain}
\thispagestyle{plain}

% -----------------------------------------------------------------------
%
% Macro f�r das Abgabeblatt
%
\newcommand{\Abgabeblatt}[8]
          {\Arbblatt{#1}{#2}{#3}{#4}{#5}{#6}{#7}{#8}{"Ubungsblatt}}

\newcommand{\grenzop}[3]{\mathop{#1}\limits^{#2}_{#3}}

\newcommand{\langrechtspfeil}[2]{\grenzop{\longrightarrow}{#1}{#2}}
\newcommand{\bob}[0]{\hspace*{\fill}$\Box$}

% -----------------------------------------------------------------------

\newcommand{\Arbblatt}[9]
{
\begin{tabular}[t]{lr}
  \begin{tabular}[t]{l}
    Mathematik I\hspace{25ex} WiSe 2016/2017\\ Tutor: {#3}\\ \hspace{75ex} \\
    {\Huge\sf {#9} {#1} } \\[1.5ex]
    {\Large Aufgabenl"osung} \\[1ex]
    {\Large Abgabe: {#2}} \\[1ex]
  \end{tabular}
  &
  \begin{tabular}[t]{l}
      {#4}\\
      {#5}\\
      {#6}\\
      {#7}\\
      {#8}\\
  \end{tabular} \\[2ex]
\end{tabular}\\
\rule{\linewidth}{1pt}
}

%%% Local Variables: 
%%% mode: latex
%%% TeX-master: t
%%% End: 


\begin{document}

\Abgabeblatt{4}{15. November 2016}{Niklas Gather \#10}
                {Lena Zamzow}{Tobias Pfannschmidt}{Daniel Zaccaria}{}{}


%%%%%%%%%%%%%%%%%%%%%%%%%%%%%%%%%%%%%%%%%%%%%%%%
%%%%%%%%%%%%%%%%%%%%%%%%%%%%%%%%%%%%%%%%%%%%%%%%

\section*{Aufgabe H8}

\subsection*{a)}

$M = \{1,2\}$ \\
$R = R \subseteq M \times M$ \\
$M \times M =   \{ (1,1), (1,2), (2,1), (2,2)\}$ \\

Im Folgenden werden alle Relationen auf M angegeben: \\
\begin{math}
\{ (1,1) \} ; \{ (1,2) \} ; \{ (2,1) \} ; \{ (2,2) \} ; \\
\{ (1,1), (1,2) \} ; \{ (1,1), (2,1) \} ; \{ (1,1), (2,2) \} ; \{ (1,2), (2,1) \} ; \{ (1,2), (2,2) \} ; \{ (2,1), (2,2) \} ; \\
\{ (1,1), (1,2), (2,1) \} ; \{ (1,1), (1,2), (2,2) \} ; \{ (1,1), (2,1), (2,2) \} ; \{ (1,2), (2,1), (2,2) \} ; \\
\{ (1,1), (1,2), (2,1), (2,2)\} \\
\end{math}

\subsection*{b)}

Folgende Relationen aus a) sind \textbf{Äquivalenzrelationen}: \\
\begin{math}
\{ (1,1) \} ; \{ (2,2) \} ; \\
\{ (1,1), (2,2) \} ; \\ 
\{ (1,1), (1,2), (2,1), (2,2)\} ; \\
\end{math}

Folgende Relationen aus a) sind \textbf{Halbordnungen}: \\
\begin{math}
\{ (1,1), (2,2) \}; \\
\{ (1,1), (1,2), (2,2) \} ; \{ (1,1), (2,1), (2,2) \} \\
\end{math}

Folgende Relationen aus a) sind \textbf{Totale Ordnungen}: \\
\begin{math}
\{ (1,1), (1,2), (2,2) \} ; \{ (1,1), (2,1), (2,2) \} \\
\end{math}

\subsection*{c)}

Folgende Relationen aus a) sind \textbf{rechtseindeutig}: \\
\begin{math}
\{ (1,1) \} ; \{ (1,2) \} ; \{ (2,1) \} ; \{ (2,2) \} ; \\
\{ (1,1), (2,2) \} ; \{ (1,2), (2,1) \} \\
\end{math}

Folgende Relationen aus a) sind \textbf{linkstotal}: \\
\begin{math}
\{ (1,1), (2,1) \} ; \{ (1,1), (2,2) \} ; \{ (1,2), (2,1) \} ; \{ (1,2), (2,2) \} ; \\
\{ (1,1), (1,2), (2,1) \} ; \{ (1,1), (1,2), (2,2) \} ; \{ (1,1), (2,1), (2,2) \} ; \{ (1,2), (2,1), (2,2) \} ; \\
\{ (1,1), (1,2), (2,1), (2,2)\} \\
\end{math}

Folgende Relationen aus a) sind \textbf{Abbildungen}: \\
\begin{math}
\{ (1,1), (2,2) \} ; \{ (1,2), (2,1) \} \\
\end{math}


\section*{Aufgabe H9}
Es soll gezeigt werden, dass \(R\) eine Äquivalenzrelation auf \(A\) ist. \(R\) ist wie folgt gegeben:
\begin{equation*}
xRy \Leftrightarrow \exists M \in P: x \in M \land y \in M
\end{equation*}

Damit \(R\) eine Äquivalenzrelation ist, muss \(R\) \textbf{reflexiv, symmetrisch} und \textbf{transitiv} sein.

\subsubsection*{reflexiv}
Die Reflexivität ist gegeben, da nach der Definition einer Partition für alle \(x \in A\) es ein \(M \in P\) gibt für das gilt \(x \in M\).
\begin{align*}
xRy & \Leftrightarrow x \in M \land y \in M \subseteq M \in P  & \text{P disjunkt, nicht-leere Menge}\\
xRy & \Leftrightarrow \bar{x} = \{x\} \subseteq M \in P \text{~und~} \bar{y} = \{y\} \subseteq M \in P & \text{Satz 3.5:~}\bar{x} = \bar{y} \\
\end{align*}
\subsubsection*{symmetrisch}
Da die Relation \(xRy\) die Bedingung hat, dass \(x \in M \land y \in M\) ergibt sich nach der Definition von Symmetrie in Relationen, dass ebenfalls gilt \(yRx\). Somit ist auch die zweite Notwendigkeit für die Äquivalenzrelation gegeben.

\begin{align*}
xRy & \Leftrightarrow x \in M \land y \in M \subseteq M \in P  \\
xRy & \Leftrightarrow \bar{x} = \bar{y} \Rightarrow xRy = yRx & \text{Satz 3.6:~} \bar{x} = \bar{y} \text{~oder~} \bar{x} \cap \bar{y} = \varnothing\\
\end{align*}

\subsubsection*{transitiv}

\begin{align*}
~ & xRy \Rightarrow x,y \in M_1 \subset M\\
\text{Wenn gilt:~} & yRz \Rightarrow y,z \in M_2 \subset M \\
\end{align*}
Da alle Elemente aus \(A\) sich auf die Mengen in \(P\) verteilen, ergibt sich, dass \(M_1 = M_2\), da beide das Element \(y\) enthalten.
\begin{align*}
\Rightarrow xRz, & \text{~somit ist~} R \text{~auch transitiv}
\end{align*}

Daraus folgt, dass $R$ reflexiv, symmetrische und transitiv ist. Daraus folgt: $R$ ist Äquivlanzrelation auf $A$.
\section*{Aufgabe H10}

\textbf{Satz 3.6.} Sei \emph{A} eine Menge und \emph{R} eine Äquivalenzrelation auf \emph{A} und $x, y \in A$. \\
Dann gilt: $ \bar{x} = \bar{y} \text{~oder~} \bar{x} \cap \bar{y} = \varnothing$ \\

Im folgenden wird der Beweis des Satzes 3.6 mit Begründungen dargelegt:

\begin{itemize}
\item[a)] Angenommen $z \in \bar{x} \cap \bar{y}$ \\
Wir möchten die Behauptung aus Satz 3.6 beweisen. Dazu treffen wir zunächst die Annahme, dass ein Element $z$ im Schnitt von $\bar{x}$ und $\bar{y}$ enthalten ist. Im Folgenden gilt es zu beweisen (unter der Voraussetzung, dass \emph{R} eine Äquivalenzrelation auf A ist und $x,y \in A$), dass dann gilt: $\bar{x} = \bar{y}$
\item[b)] Dann ist $z \in \bar{x} \land z \in \bar{y}$ \\
Begründung: Die Menge $\bar{x}$ geschnitten mit der Menge $\bar{y}$ beinhaltet gerade die Elemente, die in beiden Mengen enthalten sind. An dieser Stelle des Beweises formen wir also die Darstellung des Sachverhaltes in Mengenschreibweise in einen logischen Ausdruck um.
\item[c)] also gilt $zRx$ und $zRy$ \\
Begründung: Nach Definition 3 aus der Vorlesung sind Äquivalenzklassen folgendermaßen definiert: \emph{A} sei eine Menge und \emph{R} eine Äquivalenzrelation auf \emph{A}. Dann ist die Äquivalenzklasse von $x \in A$ : $ \bar{x} = \{ z \in A | zRx \}$. Die Äquivalenzklasse des Elementes x enhält also gerade alle Elemente $z$, die mit x ein geordnetes Paar in der Menge \emph{R} bilden, also für die gilt zRx. Daraus folgt: ist z ein Element der Äquivalenzklassen $\bar{x}$ und $\bar{y}$, so muss gelten: \emph{zRx} und \emph{zRy}.
\item[d)]  Wegen der Symmetrie und Transitivität von \emph{R} gilt dann \emph{xRy} \\
Begründung: \emph{R} ist eine Äquivalenzrelation auf \emph{A} und somit gilt, dass sie symmetrisch und transitiv ist. Aus der Symmetrie ergibt sich, dass wenn für die Relation \emph{R} zRx und zRy gilt, auch xRz und yRz gelten müssen (Def. 2, Punkt 2 Vorlesung). Aus der Transivität von \emph{R} folgt: Gelten xRz und zRy, so muss auch xRy gelten (Def. 2, Punkt 4 Vorlesung).
\item[e)] und nach Satz 5 aus Kapitel 3: $\bar{x} = \bar{y}$ \\
Begründung: In Satz 3.5 aus der Vorlesung haben wir bewiesen, dass wenn \emph{R} eine Relation auf die Menge \emph{A} ist (also: $R \subseteq A \times A$) und xRy gilt, dass $\bar{x} = \bar{y}$. Da wir in den Schritten a) - d) gezeigt haben, dass xRy gilt, folgt aus dem Beweis von Satz. 3.5 direkt, dass $\bar{x} = \bar{y}$.
\end{itemize}


\section*{Aufgabe B2}

\begin{math}
A = \{1,2,3,4,5\} \\
R = \{ (1,1), (1,4), (2,1), (2,2), (3,3), (3,5), (4,3), (4,5), (5,5) \}
\end{math}


\subsection*{a)}
Eine Halbordnung ist nach Defintion 2, Punkt 7 aus Der Vorlesung folgendermaßen definiert: \emph{R} ist eine Relation auf \emph{A}. Dann heißt $R$ Halbordnung, wenn sie \textbf{reflexiv}, \textbf{antisymmetrisch} und \textbf{transitiv} ist. Ist eine dieser Eigenschaften nicht erfüllt, ist \emph{R} keine Halbordnung. \\
\(R\) ist somit keine Halbordnung, weil:
\begin{enumerate}
\item \emph{R} ist nicht reflexiv, da \( (4,4) \notin R \)
\item \emph{R} ist antisymmetrisch (dies ist nicht Teil der Begründung dafür, warum \emph{R} keine Halbordnung ist, sondern wird nur der Vollständigkeit halber aufgeführt) \\
(Folgende Elemente, die es bei Antisymmetrie von \emph{R} nicht geben darf, sind nicht enthalten: \((4,1),(1,2),(5,3),(3,4),(5,4)\) )
\item \emph{R} ist nicht transitiv, weil z.B. \( (1,4) \in R \land (4,5) \in R \) aber \((1,5) \notin R\)
\end{enumerate}

\subsection*{b)}
Damit \(R\) eine Halbordnung wird müssen mindestens die Elemente ergänzt werden, die für die Erfüllung der Reflexivität und Transitivität benötigt werden. Durch Hinzufügen von Elementen erhalten wir eine neue Relation $R'$.
\subsubsection*{Reflexivität}
    Das Element \((4,4)\) muss hinzugefügt werden, weil \(xRx\) gelten muss und  \(4 \in A\).

\subsubsection*{Transitivität}
    Das Element \((1,5)\) muss hinzugefügt werden, weil $ (1,4) \in R \land (4,5) \in R$.\\
    Das Element \((1,3)\) muss hinzugefügt werden, weil $ (1,4) \in R \land (4,3) \in R$.\\
    Das Element \((2,4)\) muss hinzugefügt werden, weil $ (2,1) \in R \land (1,4) \in R $.\\
    Das Element \((2,3)\) muss hinzugefügt werden, weil $ (2,1) \in R \land (1,3) \in R' $.\\
    Das Element \((2,5)\) muss hinzugefügt werden, weil $ (2,3) \in R' \land (3,5) \in R $.\\

Daraus ergibt sich folgende Relation $R'$, die eine Halbordnung ist:\\
\begin{equation*}
  R' = \{(1,1), \mathbf{(1,3)},(1,4), \mathbf{(1,5)}, (2,2), \mathbf{(2,3), (2,4), (2,5)}, (3,3), (3,5), (3,5), (4,3), \mathbf{(4,4)},(4,5), (5,5)\}
\end{equation*}

\subsection*{c)}
Die unter b) erhaltene Halbordnung $R'$ ist eine Totale Ordnung. \\

Damit $R'$ total ist, muss für alle $x, y \in A$ gelten: \(xRy\) oder \(yRx\).
\begin{align*}
\mathbf{(1,3)} & \lor (3,1) \\ 
\mathbf{(1,4)} & \lor (4,1) \\ 
\mathbf{(1,5)} & \lor (5,1) \\ 
\mathbf{(2,1)} & \lor (1,2) \\ 
\mathbf{(2,3)} & \lor (3,2) \\ 
\mathbf{(2,4)} & \lor (4,2) \\ 
\mathbf{(2,5)} & \lor (5,2) \\ 
\mathbf{(3,4)} & \lor (4,3) \\ 
\mathbf{(3,5)} & \lor (5,3) \\ 
\mathbf{(4,5)} & \lor (5,4) \\ 
\end{align*}
Die fett markierten geordneten Paare sind Elemente von $R'$, damit ist $R'$ total, daraus folgt: $R'$ ist eine Totale Ordnung.


\end{document}
