\documentclass[11pt,a4paper]{article}

\usepackage{german}
\usepackage{amssymb}
\usepackage{amsmath}
\usepackage{epsfig}
\usepackage{graphics}
\usepackage{moreverb}
\usepackage[utf8]{inputenc}
% -----------------------------------------------------------------------
%
% Page layout
%

% gieza bigger page
\setlength{\textwidth}{16cm}
\setlength{\topmargin}{-1cm}
\setlength{\evensidemargin}{0cm}
\setlength{\oddsidemargin}{0cm}
\setlength{\textheight}{24cm}

% skip between paragraphs
\setlength{\parskip}{1ex}
% ... and no indentation at start of a new paragraph
\setlength{\parindent}{0ex}

\pagestyle{plain}
\thispagestyle{plain}

% -----------------------------------------------------------------------
%
% Macro f�r das Abgabeblatt
%
\newcommand{\Abgabeblatt}[8]
          {\Arbblatt{#1}{#2}{#3}{#4}{#5}{#6}{#7}{#8}{"Ubungsblatt}}

\newcommand{\grenzop}[3]{\mathop{#1}\limits^{#2}_{#3}}

\newcommand{\langrechtspfeil}[2]{\grenzop{\longrightarrow}{#1}{#2}}
\newcommand{\bob}[0]{\hspace*{\fill}$\Box$}

% -----------------------------------------------------------------------

\newcommand{\Arbblatt}[9]
{
\begin{tabular}[t]{lr}
  \begin{tabular}[t]{l}
    Mathematik I\hspace{25ex} WiSe 2016/2017\\ Tutor: {#3}\\ \hspace{75ex} \\
    {\Huge\sf {#9} {#1} } \\[1.5ex]
    {\Large Aufgabenl"osung} \\[1ex]
    {\Large Abgabe: {#2}} \\[1ex]
  \end{tabular}
  &
  \begin{tabular}[t]{l}
      {#4}\\
      {#5}\\
      {#6}\\
      {#7}\\
      {#8}\\
  \end{tabular} \\[2ex]
\end{tabular}\\
\rule{\linewidth}{1pt}
}

%%% Local Variables: 
%%% mode: latex
%%% TeX-master: t
%%% End: 

\begin{document}

\Abgabeblatt{1}{07.03.2017}{Tim Haga}
                {Lena Zamzow}{Tobias Pfannschmidt}{}{}{}


%%%%%%%%%%%%%%%%%%%%%%%%%%%%%%%%%%%%%%%%%%%%%%%%
%%%%%%%%%%%%%%%%%%%%%%%%%%%%%%%%%%%%%%%%%%%%%%%%
\section*{Eckdaten}
\begin{itemize}
\subsection*{Umfang}
	\item Zeit: 120 Minuten
	\item Aufgaben: 6 + 1 Bonusaufgabe
	\item Punkte: 70 + 10 (Teilaufgaben entsprechend 5 oder 10 Punkte)
	\item 2-3 \emph{leichte} Aufgaben	
\end{itemize}
\begin{itemize}
\subsection*{Vorbereitung}
	\item Skript, Notizen, Übungen rekapitulieren
	\item Aufgaben erneut durchrechnen (beste Vorbereitung)
	\item positive Grundstimmung entwickeln
\end{itemize}
\begin{itemize}
	\subsection*{Tipps}
	\item chronologische Aufgabenstellung (ältere Themen zuerst) 
	\item 5 Minuten zu Beginn Zeit nehmen und die gesamte Klausur anschauen, beginnen mit den Aufgaben die wir am besten können
	\item Lösungen nachvollziehbar machen (Schrittweise Vorgehen)
	\item Beweise (nur chin. Restesatz), Sätze, Lemma auswendig
	\item Unterschiedliche Themen werden dran kommen (nicht zuviel Gedanken um Geometrie machen: Was ist die Basis von $\cdots$?, Skalarprodukt)
\end{itemize}

\newpage

\section*{Kapitel 1: Logik}
\begin{itemize}
	\item kein Chomp und Gewinnstrategie
	\item Logik: Aussage, Aussageform, Wahrheitstafel, logischer Term, Negation, Rechenregeln, Tautologie, Kontradiktion,   Ausdruck DNF und KNF, DNF und KNF logisch äquivalent, neuen logischen Operator und Ausdruck angeben (keine formale Schlussregel)
	\begin{center}
		\begin{tabular}{ | l | l | l | }
			\hline
			Logik& \\ \hline	\hline	
			Definition&1,7,13,14,17,19,22 \\ \hline
			Notation&9 \\ \hline
			Satz&16,24\\
			\hline	
		\end{tabular}
	\end{center}
\end{itemize}
	
\section*{Kapitel 2: Mengen}
\begin{itemize}
	\item Mengenoperationen, grundlegende Definitionen, nicht alle Axiome
	\begin{center}
		\begin{tabular}{ | l | l | l | }
			\hline
			Mengen& \\ \hline	\hline	
			Definition&1,7,9,13,15,20,22,25,28,29 \\ \hline
			Axiome&4,6,16 \\ \hline
			Notation&11 \\ \hline
			Satz&23\\ 
			\hline	
		\end{tabular}
	\end{center}	
\end{itemize}

\section*{Kapitel 3: Relationen und Abbildungen}
\begin{itemize}
	\item Was ist eine Relation?, Äquivalenzrelation, Äquivalenzklasse Bild und Urbild, injektiv, surjektiv, bijektiv
	\begin{center}
		\begin{tabular}{ | l | l | l | }
			\hline
			Relationen und Abbildungen& \\ \hline	\hline	
			Definition&1,2,12,13,17,22 \\ \hline
			Satz&5,10,11,25(5 Eigenschaften von Verknüpfungen)\\ 
			\hline	
		\end{tabular}
	\end{center}	
\end{itemize}

\section*{Kapitel 4: Natürliche Zahlen und Vollständige Induktion}
\begin{itemize}
	\item nicht Kapitel \emph{4.2 Axiomatisierung der natürlichen Zahlen (Peamo-Axiom)}
	\item a $\mid$ b oder a $\nmid$ b, m $\mid$ 0 und p als Primzahl hat nur triviale Teiler
	\begin{center}
		\begin{tabular}{ | l | l | l | }
			\hline
			Natürliche Zahlen& \\ \hline	\hline	
			Definition&2,9\\ \hline
			Notation& 1\\ \hline
			Beispiel& 3\\ 
			\hline	
		\end{tabular}
	\end{center}	
\end{itemize}

\newpage

\section*{Kapitel 5: Zahlentheorie (!wichtig!)}
\begin{itemize}
	\item Euklidischer und erweiterter euklidischer Algorithmus kommt dran
	\item Chinesischer Restesatz (mit Beweis) kommt dran
	\begin{center}
		\begin{tabular}{ | l | l | l | }
			\hline
			Zahlentheorie& \\ \hline	\hline	
			Definition&4(Betrag), 5(a Teiler b), 10(Kongruenz),12,16(ggT),28,34(ggT von mehr als 2 Z.)\\ \hline
			Proposition&6(Halbordnung auf n ($\rightarrow$ Aussage aber nicht den Beweis)\\ \hline
			Lemma&8(Division mit Rest),17,22,24(mod kürzen),25(ohne ISBN)\\ \hline
			Notation&9(Modulo/ggT zweier Zahlen)\\ \hline
			Satz&\textbf{18+20(Euklid), 32+35(paarw. teilerfremd)(Chinesischer Restesatz Bew.)}\\ 
			\hline	
		\end{tabular}
	\end{center}	
\end{itemize}

\section*{Kapitel 6: Kombinatorik}
\begin{itemize}
	\item Schubfachprinzip und Prinzip Inklusion-Exklusion 70\%ige Wahrscheinlichkeit 
	\begin{center}
		\begin{tabular}{ | l | l | l | }
			\hline
			Kombinatorik& \\ \hline	\hline	
			Definition&1,3 (binomischer Lehrsatz),6(Fakultät)\\ \hline
			Satz& 5,7,\textbf{10(Schubfachprinzip), 19(Inklusion-Exklusion)}\\ \hline
			Korollar&11(keine injektive Abb. wenn eine Menge größer ist als die andere) \\ 
			\hline	
		\end{tabular}
	\end{center}	
\end{itemize}

\section*{Kapitel 7: Geometrie}
\begin{itemize}
	\item Keine Informationen zu dem Kapitel
	\begin{center}
		\begin{tabular}{ | l | l | l | }
			\hline
			\textbf{Geometrie}& \\ \hline	\hline	
			Definition&1,2,3,4,7,9,10,11,12,16,18,23,29\\ \hline
			\hline	
		\end{tabular}
	\end{center}	
\end{itemize}

\section*{Kapitel 8: Gruppentheorie (größtes!)}
\begin{itemize}
	\item nicht Kryptografie
	\item Beweis Gruppenhomomorphismus (Def. 17) und Satz von Lagrange (Satz 29) auswendig wissen $\rightarrow$ kommt dran
	\begin{center}
		\begin{tabular}{ | l | l | l | }
			\hline
			\textbf{Gruppentheorie}& \\ \hline	\hline	
			Definition&\textbf{1,3},4,6,10,12,14,15,\textbf{17(Beweis)},24,27,31,32,37,40,41(Beweis),42\\ \hline
			Satz& \textbf{29(Lagrange)}\\
			\hline	
		\end{tabular}
	\end{center}	
\end{itemize}

\section*{Kapitel 9: Graphen}
\begin{itemize}
	\item nicht Sprouds und Broussel Sprouds
	\begin{center}
		\begin{tabular}{ | l | l | l | }
			\hline
			\textbf{Graphen}& \\ \hline	\hline	
			Definition&1,4,6,8,12,13,14,17,19,26\\ \hline
			Korollar&18 (letztes Jahr in der Klausur) \\ \hline
			Satz& 11,21, \textbf{28(Euler-Formel)}\\ 
			\hline	
		\end{tabular}
	\end{center}	
\end{itemize}

\section*{Übungsaufgaben}

	\begin{center}
		\begin{tabular}{ | l | l | l | }
			\hline
			&\textbf{Präsens} \\ \hline	\hline	
			Aufgaben&4,5,6,7,11,15,16,17,18,19,21,22,23,24,25,28(letztes Jahr Klausur),29,31,32\\
			\hline	
		\end{tabular}
	\end{center}
	\begin{center}
		\begin{tabular}{ | l | l | l | }
			\hline
			&\textbf{Haus}\\ \hline	\hline	
			Aufgaben&3,4,5,9,11,13,16,18,21,22,23,24,25,26\\
			\hline	
		\end{tabular}
	\end{center}	
	\begin{center}
		\begin{tabular}{ | l | l | l | }
			\hline
			&\textbf{Weihnachtsblatt} \\ \hline	\hline	
			Aufgaben&2\\
			\hline	
		\end{tabular}
	\end{center}
	\begin{center}
		\begin{tabular}{ | l | l | l | }
			\hline
			&\textbf{Benotete} \\ \hline	\hline	
			Aufgaben&2,3,5,6\\
			\hline	
		\end{tabular}
	\end{center}

\end{document}
