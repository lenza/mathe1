\documentclass[11pt,a4paper]{article}

\usepackage{german}
\usepackage{amssymb}
\usepackage{amsmath}
\usepackage{epsfig}
\usepackage{graphics}
\usepackage{moreverb}
\usepackage[utf8]{inputenc}
% -----------------------------------------------------------------------
%
% Page layout
%

% gieza bigger page
\setlength{\textwidth}{16cm}
\setlength{\topmargin}{-1cm}
\setlength{\evensidemargin}{0cm}
\setlength{\oddsidemargin}{0cm}
\setlength{\textheight}{24cm}

% skip between paragraphs
\setlength{\parskip}{1ex}
% ... and no indentation at start of a new paragraph
\setlength{\parindent}{0ex}

\pagestyle{plain}
\thispagestyle{plain}

% -----------------------------------------------------------------------
%
% Macro f�r das Abgabeblatt
%
\newcommand{\Abgabeblatt}[8]
          {\Arbblatt{#1}{#2}{#3}{#4}{#5}{#6}{#7}{#8}{"Ubungsblatt}}

\newcommand{\grenzop}[3]{\mathop{#1}\limits^{#2}_{#3}}

\newcommand{\langrechtspfeil}[2]{\grenzop{\longrightarrow}{#1}{#2}}
\newcommand{\bob}[0]{\hspace*{\fill}$\Box$}

% -----------------------------------------------------------------------

\newcommand{\Arbblatt}[9]
{
\begin{tabular}[t]{lr}
  \begin{tabular}[t]{l}
    Mathematik I\hspace{25ex} WiSe 2016/2017\\ Tutor: {#3}\\ \hspace{75ex} \\
    {\Huge\sf {#9} {#1} } \\[1.5ex]
    {\Large Aufgabenl"osung} \\[1ex]
    {\Large Abgabe: {#2}} \\[1ex]
  \end{tabular}
  &
  \begin{tabular}[t]{l}
      {#4}\\
      {#5}\\
      {#6}\\
      {#7}\\
      {#8}\\
  \end{tabular} \\[2ex]
\end{tabular}\\
\rule{\linewidth}{1pt}
}

%%% Local Variables: 
%%% mode: latex
%%% TeX-master: t
%%% End: 

\begin{document}

\Abgabeblatt{10}{17. Januar 2017}{Niklas Gather \#10}
                {Lena Zamzow}{Tobias Pfannschmidt}{}{}{}


%%%%%%%%%%%%%%%%%%%%%%%%%%%%%%%%%%%%%%%%%%%%%%%%
%%%%%%%%%%%%%%%%%%%%%%%%%%%%%%%%%%%%%%%%%%%%%%%%
\section*{Aufgabe H22}
Multiplikationstabelle der $\mathbb{Z}^{*}_{20} = \{1,3,7,9,11,13,17,19\}$ Ordnung:8\\
\begin{center}
	\begin{tabular}{ | l | l | l | l | l | l | l| l|l|}
		\hline
		$\cdot$&1&3&7&9&11&13&17&19 \\ \hline \hline		
		1&1&3&7&9&11&13&17&19 \\ \hline 	
		3&3&9&1&7&13&19&11&17 \\ \hline
		7&7&1&9&3&17&11&19&13\\ \hline	
		9&9&7&3&1&19&17&13&11 \\ \hline
		11&11&13&17&19&1&3&7&9 \\ \hline
		13&13&19&11&17&3&9&1&7 \\ \hline
		17&17&11&19&13&7&1&9&3 \\ \hline
		19&19&17&13&11&9&7&3&1 \\ \hline 
	\end{tabular}
\end{center}
\section*{Aufgabe H23}
\subsection*{a}
Beh.: Sind e und f neutrale Elemente einer Gruppe G, dann gilt e = f. \\

Da e und f neutrale Elemente sind gilt: \\
$e \ast g = g = g \ast e$ \\
$f \ast g = g = g \ast e$ \\
Außerdem gilt: \\
$g \ast g^{-1} = e$ und $g \ast g^{-1} = f$ \\

z.z.: e = f \\
$ e = e \ast g = (g \ast g^{-1}) \ast  g = f \ast  g = f ~~~_\Box$

\subsection*{b}
<<<<<<< Updated upstream
\section*{Aufgabe B6}
Die Untergruppen zu der Gruppe $\mathbb{Z}^{*}_{20} = \{1,3,7,9,11,13,17,19\}$ sind:\\
Die trivialen Untergruppen:\\
$U_1 =\{1\} \longrightarrow$ neutrales Element\\
$U_20 = \{1,3,7,9,11,13,17,19\}$\\
Inverse Elemente:\\
$3^{-1}=7, 9^{-1}=9, 11^{-1}=11, 13^{-1}=17, 19^{-1}=19$\\
$U_3=\{1,9\}$\\
$U_4=\{1,11\}$\\
$U_5=\{1,19\}$\\
$U_6=\{1,9,11,19\}$\\
$U_7=\{1,3,7,9\}$\\
$U_8=\{1,9,13,17\}$\\
=======
Beh.: Sind h und k beides inverse Elemente zu einem Gruppenelement g $\in$ G, so gilt h = k. \\

Da h und k inverse Elemente zu g sind (linksinvers und rechtsinvers, wie vorher bewiesen) gilt: \\
$ h \ast  g = e = g \ast  h$ \\
$ k \ast  g = e = g \ast  k$  \\

z.z.: h = k \\
$h = h \ast e$   (zu jedem Element gibt es ein linksneutrales Element) \\
$= h \ast (g \ast k)$    (siehe oben) \\
$= (h \ast g) \ast k$    (Assoziativität) \\
$= e \ast k$      (siehe oben) \\
$=k$       (neutrales Element) $~~~_\Box$ \\

\section*{Aufgabe B6}

>>>>>>> Stashed changes
\end{document}
