\documentclass[11pt,a4paper]{article}

\usepackage{german}
\usepackage{amssymb}
\usepackage{amsmath}
\usepackage{epsfig}
\usepackage{graphics}
\usepackage{moreverb}
\usepackage[utf8]{inputenc}
% -----------------------------------------------------------------------
%
% Page layout
%

% gieza bigger page
\setlength{\textwidth}{16cm}
\setlength{\topmargin}{-1cm}
\setlength{\evensidemargin}{0cm}
\setlength{\oddsidemargin}{0cm}
\setlength{\textheight}{24cm}

% skip between paragraphs
\setlength{\parskip}{1ex}
% ... and no indentation at start of a new paragraph
\setlength{\parindent}{0ex}

\pagestyle{plain}
\thispagestyle{plain}

% -----------------------------------------------------------------------
%
% Macro f�r das Abgabeblatt
%
\newcommand{\Abgabeblatt}[8]
          {\Arbblatt{#1}{#2}{#3}{#4}{#5}{#6}{#7}{#8}{"Ubungsblatt}}

\newcommand{\grenzop}[3]{\mathop{#1}\limits^{#2}_{#3}}

\newcommand{\langrechtspfeil}[2]{\grenzop{\longrightarrow}{#1}{#2}}
\newcommand{\bob}[0]{\hspace*{\fill}$\Box$}

% -----------------------------------------------------------------------

\newcommand{\Arbblatt}[9]
{
\begin{tabular}[t]{lr}
  \begin{tabular}[t]{l}
    Mathematik I\hspace{25ex} WiSe 2016/2017\\ Tutor: {#3}\\ \hspace{75ex} \\
    {\Huge\sf {#9} {#1} } \\[1.5ex]
    {\Large Aufgabenl"osung} \\[1ex]
    {\Large Abgabe: {#2}} \\[1ex]
  \end{tabular}
  &
  \begin{tabular}[t]{l}
      {#4}\\
      {#5}\\
      {#6}\\
      {#7}\\
      {#8}\\
  \end{tabular} \\[2ex]
\end{tabular}\\
\rule{\linewidth}{1pt}
}

%%% Local Variables: 
%%% mode: latex
%%% TeX-master: t
%%% End: 

\begin{document}

\Abgabeblatt{9}{19. Dezember 2016}{Niklas Gather \#10}
                {Lena Zamzow}{Tobias Pfannschmidt}{}{}{}


%%%%%%%%%%%%%%%%%%%%%%%%%%%%%%%%%%%%%%%%%%%%%%%%
%%%%%%%%%%%%%%%%%%%%%%%%%%%%%%%%%%%%%%%%%%%%%%%%
\section*{Aufgabe H21}
\subsection*{a}
Behauptung: $\forall~n,k \in \mathbb{N} k \leq n $ gilt:$ (
\begin{smallmatrix}
n\\k
\end{smallmatrix} ) = \frac{n!}{k!(n-k)!}$\\
Beweis durch vollständige Induktion über n.\\ \\
IA: $n=0\\
(
\begin{smallmatrix}
0\\0
\end{smallmatrix} ) = \frac{0!}{0!(0-0)!}\\
1=\frac{1}{1\cdot 1}\\
1=1 \checkmark$
\\ \\
IV: $(
\begin{smallmatrix}
	n\\k
\end{smallmatrix} ) = \frac{n!}{k!(n-k)!}$\\
\\ \\
IS: $(
\begin{smallmatrix}
n+1\\k
\end{smallmatrix} ) = 
\frac{(n+1)!}{k!((n+1)-k)!}\\= (
\begin{smallmatrix}
n\\k-1
\end{smallmatrix} ) + (
\begin{smallmatrix}
n\\k 
\end{smallmatrix} ) \\
= $(Nach IV)
\\
$(
\begin{smallmatrix}
	n\\k-1
\end{smallmatrix} ) + \frac{n!}{k!(n-k)!}$\\
\\ \\
Fallunterscheidung:\\
1.Fall: $k<n$\\
=(Nach IV)$ \\
\frac{n!}{(n-1)!\cdot (n-(k-1))!} + \frac{n!}{k!(n-k)!}
$
\\ \\
2.Fall: $k=n\\ (
\begin{smallmatrix}
	n\\n-1
\end{smallmatrix} ) + 
\frac{n!}{k!(n-k)!}\\ 
= (\begin{smallmatrix}
n\\n-1
\end{smallmatrix} ) + 
\frac{n!}{k!0!}\\ = (
\begin{smallmatrix}
n\\n-1
\end{smallmatrix} ) + 
\frac{n!}{k!}\\ 
= ( \begin{smallmatrix}
n\\n-1
\end{smallmatrix} ) + 1$\\

\subsection*{b}

\section*{B5}
Unter neun beliebigen natürlichen Zahlen gibt es stets zwei Zahlen $a,b$, so dass gilt: $8|(a-b)$.\\
Als Kongruenzgleichung:
$a \equiv b (mod 8)$\\
s.d. $a (mod 8) = b (mod 8)$\\ \\
Unsere Zahlen $a,b$ müssen beim gleichen ggT den gleichen Rest lassen. Bei neun beliebigen natürlichen Zahlen muss es immer zwei geben, die die gleiche Restklasse haben.

\end{document}
