\documentclass[11pt,a4paper]{article}

\usepackage{german}
\usepackage{amssymb}
\usepackage{amsmath}
\usepackage{epsfig}
\usepackage{graphics}
\usepackage{moreverb}
\usepackage[utf8]{inputenc}
% -----------------------------------------------------------------------
%
% Page layout
%

% gieza bigger page
\setlength{\textwidth}{16cm}
\setlength{\topmargin}{-1cm}
\setlength{\evensidemargin}{0cm}
\setlength{\oddsidemargin}{0cm}
\setlength{\textheight}{24cm}

% skip between paragraphs
\setlength{\parskip}{1ex}
% ... and no indentation at start of a new paragraph
\setlength{\parindent}{0ex}

\pagestyle{plain}
\thispagestyle{plain}

% -----------------------------------------------------------------------
%
% Macro f�r das Abgabeblatt
%
\newcommand{\Abgabeblatt}[8]
          {\Arbblatt{#1}{#2}{#3}{#4}{#5}{#6}{#7}{#8}{"Ubungsblatt}}

\newcommand{\grenzop}[3]{\mathop{#1}\limits^{#2}_{#3}}

\newcommand{\langrechtspfeil}[2]{\grenzop{\longrightarrow}{#1}{#2}}
\newcommand{\bob}[0]{\hspace*{\fill}$\Box$}

% -----------------------------------------------------------------------

\newcommand{\Arbblatt}[9]
{
\begin{tabular}[t]{lr}
  \begin{tabular}[t]{l}
    Mathematik I\hspace{25ex} WiSe 2016/2017\\ Tutor: {#3}\\ \hspace{75ex} \\
    {\Huge\sf {#9} {#1} } \\[1.5ex]
    {\Large Aufgabenl"osung} \\[1ex]
    {\Large Abgabe: {#2}} \\[1ex]
  \end{tabular}
  &
  \begin{tabular}[t]{l}
      {#4}\\
      {#5}\\
      {#6}\\
      {#7}\\
      {#8}\\
  \end{tabular} \\[2ex]
\end{tabular}\\
\rule{\linewidth}{1pt}
}

%%% Local Variables: 
%%% mode: latex
%%% TeX-master: t
%%% End: 

\begin{document}

\Abgabeblatt{9}{19. Dezember 2016}{Niklas Gather \#10}
                {Lena Zamzow}{Tobias Pfannschmidt}{}{}{}


%%%%%%%%%%%%%%%%%%%%%%%%%%%%%%%%%%%%%%%%%%%%%%%%
%%%%%%%%%%%%%%%%%%%%%%%%%%%%%%%%%%%%%%%%%%%%%%%%
\section*{Aufgabe H21}
\subsection*{a}
Behauptung: $\forall~n,k \in \mathbb{N} k \leq n $ gilt:$ (
\begin{smallmatrix}
n\\k
\end{smallmatrix} ) = \frac{n!}{k!(n-k)!}$\\
Beweis durch vollständige Induktion über n.\\ \\
IA: $n=0\\
(
\begin{smallmatrix}
0\\0
\end{smallmatrix} ) = \frac{0!}{0!(0-0)!}\\
1=\frac{1}{1\cdot 1}\\
1=1 \checkmark$
\\ \\
IV: $(
\begin{smallmatrix}
	n\\k
\end{smallmatrix} ) = \frac{n!}{k!(n-k)!}$\\
\\ \\
IS: $(
\begin{smallmatrix}
n+1\\k
\end{smallmatrix} ) = 
\frac{(n+1)!}{k!((n+1)-k)!}\\= (
\begin{smallmatrix}
n\\k-1
\end{smallmatrix} ) + (
\begin{smallmatrix}
n\\k 
\end{smallmatrix} ) \\
= $(Nach IV)
\\
$(
\begin{smallmatrix}
\frac{n!}{(n-1)!\cdot (n-(k-1))!} 
\end{smallmatrix} ) + \frac{n!}{k!(n-k)!}
\\ \\
Fallunterscheidung:\\
1.Fall: $k<n$\\
$
\\ \\
2.Fall: $k=n\\ (
\frac{n!}{(n-1)!\cdot (n-(k-1))!} 
\begin{smallmatrix}
+ \frac{n!}{k!(n-k)!}
\end{smallmatrix} ) + 
\frac{n!}{k!(n-k)!}\\ 
= (\begin{smallmatrix}
n\\n-1
\end{smallmatrix} ) + 
\frac{n!}{k!0!}\\ = (
\begin{smallmatrix}
n\\n-1
\end{smallmatrix} ) + 
\frac{n!}{k!}\\ 
= ( \begin{smallmatrix}
n\\n-1
\end{smallmatrix} ) + 1$\\

\subsection*{b}

\section*{B5}
\textbf{Behauptung}: Unter neun beliebigen natürlichen Zahlen gibt es stets zwei Zahlen $a,b$, sodass gilt: $8|(a-b)$.\\

\textbf{Beweis}: \\

Nach Definition 5.28 aus der Vorlesung gilt: \\

Für $a, b \in \mathbb{Z}$ und $n \in \mathbb{N}$ gibt es die Äquivalenzrelation: 
$a \sim_n b \Leftrightarrow n | (a-b)$ \\
Wir bezeichnen mit $[a]_n$ für $a \in \mathbb{Z}$ die Äquivalenzklassen von a bezüglich $\sim_n$. \\
Dabei gilt für $0 \leq a < n: [a]_n = \{x \in \mathbb{Z} | x \equiv a (mod n)\}$.

Für n = 8 gibt es nach Definition die Äquivalenzklassen $[0]_8,\cdots,[7]_8$. \\
Gemäß Beispiel 5.29 aus der Vorlesung lässt sich leicht folgern, dass die Menge der ganzen Zahlen $\mathbb{Z}$ sich in \textbf{8 Partitionen} bezüglich $\sim_8$ aufteilen lässt, sodass man schreiben kann: \\

$\mathbb{Z}_8 = \{[0]_8, [1]_8, [2]_8, [3]_8, [4]_8, [5]_8, [6]_8, [7]_8\}$ \\

Nach Def. 5.29 gilt damit: \\

$a \sim_n b \Leftrightarrow 8 | (a-b)$. \\

Das bedeutet, dass wenn man zwei Zahlen a und b aus der gleichen Äquivalenzklasse bezüglich $\sim_8$ subtrahiert, das Ergebnis durch 8 teilbar ist. Damit das Ergebnis der Subtraktion zweier Zahlen a und b \textbf{nicht} durch 8 teilbar ist, muss man diese also aus verschiedenen Äquivalenzklassen bezüglich $\sim_8$ wählen. Wählt man dagegen zwei beliebige Zahlen aus derselben Äquivalenzklasse, ist das Ergebnis der Subtraktion dieser beiden Zahlen immer durch 8 teilbar. \\

Oben haben wir bereits gezeigt, dass es genau n = 8 Äquivalenzklassen bezüglich $\sim_8$ für die Menge der ganzen Zahlen $\mathbb{Z}$ gibt. Würden wir also n = 8 natürliche Zahlen betrachten, so könnte man diese so wählen, dass sie alle genau in unterschiedlichen Äquivalenzklassen bezüglich $\sim_8$ liegen. \\
Wählen wir jedoch n+1 = 9 Zahlen für $n \in \mathbb{N}$, so müssen wir aus einer Partition, also einer Äquivalenzklasse bezüglich $\sim_8$, zwei Zahlen a und b auswählen. Für diese beiden Zahlen a,b gilt gemäß 5.29: $8 | (a-b)$.    $_\Box$


%Als Kongruenzgleichung:
%$a \equiv b (mod 8)$\\
%s.d. $a (mod 8) = b (mod 8)$\\ \\
%Unsere Zahlen $a,b$ müssen beim gleichen ggT den gleichen Rest lassen. Bei neun beliebigen natürlichen Zahlen muss es immer zwei geben, die die gleiche Restklasse haben.

\end{document}
