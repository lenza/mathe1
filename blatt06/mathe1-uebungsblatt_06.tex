\documentclass[11pt,a4paper]{article}

\usepackage{german}
\usepackage{amssymb}
\usepackage{amsmath}
\usepackage{epsfig}
\usepackage{graphics}
\usepackage{moreverb}
\usepackage[utf8]{inputenc}
% -----------------------------------------------------------------------
%
% Page layout
%

% gieza bigger page
\setlength{\textwidth}{16cm}
\setlength{\topmargin}{-1cm}
\setlength{\evensidemargin}{0cm}
\setlength{\oddsidemargin}{0cm}
\setlength{\textheight}{24cm}

% skip between paragraphs
\setlength{\parskip}{1ex}
% ... and no indentation at start of a new paragraph
\setlength{\parindent}{0ex}

\pagestyle{plain}
\thispagestyle{plain}

% -----------------------------------------------------------------------
%
% Macro f�r das Abgabeblatt
%
\newcommand{\Abgabeblatt}[8]
          {\Arbblatt{#1}{#2}{#3}{#4}{#5}{#6}{#7}{#8}{"Ubungsblatt}}

\newcommand{\grenzop}[3]{\mathop{#1}\limits^{#2}_{#3}}

\newcommand{\langrechtspfeil}[2]{\grenzop{\longrightarrow}{#1}{#2}}
\newcommand{\bob}[0]{\hspace*{\fill}$\Box$}

% -----------------------------------------------------------------------

\newcommand{\Arbblatt}[9]
{
\begin{tabular}[t]{lr}
  \begin{tabular}[t]{l}
    Mathematik I\hspace{25ex} WiSe 2016/2017\\ Tutor: {#3}\\ \hspace{75ex} \\
    {\Huge\sf {#9} {#1} } \\[1.5ex]
    {\Large Aufgabenl"osung} \\[1ex]
    {\Large Abgabe: {#2}} \\[1ex]
  \end{tabular}
  &
  \begin{tabular}[t]{l}
      {#4}\\
      {#5}\\
      {#6}\\
      {#7}\\
      {#8}\\
  \end{tabular} \\[2ex]
\end{tabular}\\
\rule{\linewidth}{1pt}
}

%%% Local Variables: 
%%% mode: latex
%%% TeX-master: t
%%% End: 

\begin{document}

\Abgabeblatt{6}{29. November 2016}{Niklas Gather \#10}
                {Lena Zamzow}{Tobias Pfannschmidt}{Daniel Zaccaria}{}{}


%%%%%%%%%%%%%%%%%%%%%%%%%%%%%%%%%%%%%%%%%%%%%%%%
%%%%%%%%%%%%%%%%%%%%%%%%%%%%%%%%%%%%%%%%%%%%%%%%
\section*{Aufgabe H14}
Die Subtraktion der ganzen Zahlen $\mathbb{Z}$ ist wie folgt definiert: \\
\begin{align*}
&\overline{(a,b)} - \overline{(c,d)} := \overline{(a,b)} + \overline{(d,c)} \\
\end{align*}


z.Z.: 
$ (a,b) \sim (a',b') \land (c, d) \sim (c',d') \Rightarrow \overline{(a,b)} - \overline{(c,d)} = \overline{(a',b')} - \overline{(c',d')}$ \\
\\
$(a,b) \sim (a',b') \Rightarrow a + b' = a' + b$ \\
$(c,d) \sim (c',d') \Rightarrow c + d' = c' + d$ \\
\\
Also gilt auch: 
$ a + b' + c' + d = c + d' + a' + b$ \\
\\
Mit dem Kommutativgesetz der Addition der natürlichen Zahlen $\mathbb{N}$ und dem Assoziativgesetz der Addition aus $\mathbb{N}$ können wir die Summanden umstellen und dann beliebig klammern, sodass wir schreiben können: \\
\begin{align*}
&(a+d) + (b'+c') = (a'+d') + (b+c) \Rightarrow (a+d, b+c) \sim (a'+d', b'+c') \\
\end{align*}

Aus dieser Relation folgt, dass die Äquivalenzklassen $\overline{(a+d,b+c)}$ und $\overline{(a'+d',b'+c')}$ gleich sind. Dies können wir nun verwenden: \\
\begin{align*}
&\overline{(a,b)} - \overline{(c,d)} \\
=& \overline{(a,b)} + \overline{(d,c)}   \tag{Definition Subtraktion aus $\mathbb{Z}$} \\
=& \overline{(a+d,b+c)}   \tag{Definition Addition $\mathbb{Z}$}\\
=& \overline{(a'+d',b'+c')}   \tag{siehe oben} \\
=& \overline{(a',b')} + \overline{(d',c')}   \tag{q.e.d.}
\end{align*}

\newpage

\section*{Aufgabe H15}
Distributivgesetz für ganze Zahlen: \\

Für alle $\overline{(a,b)}, \overline{(c,d)}, \overline{(e,f)} \in \mathbb{Z}$ gilt:\\
\begin{align*}
& (\overline{a,b}) \cdot ( \overline{(c,d)} + \overline{(e,f)} ) = \overline{(a,b)} \cdot \overline{(c,d)} +\overline{(a,b)} \cdot \overline{(e,f)}\\
\end{align*}


Beweis:
\begin{align*}
& \overline{(a,b)} \cdot ( \overline{(c,d)} + \overline{(e,f)} )\\
& = \overline{(a,b)} \cdot \overline{(c+e,d+f)}   \tag{Addition $\mathbb{Z}$}\\
& = \overline{((a \cdot (c+e)) + (b \cdot (d+f)), (a \cdot (d+f))+ (b \cdot (c+e))}   \tag{Multiplikation $\mathbb{Z}$} \\
& = \overline{((ac+ae+bd+bf),(ad+af+bc+be))}   \tag{Distributivgesetz nach Korollar 4.22}\\
& = \overline{((ac+bd+ae+bf),(ad+bc+af+be))}  \tag{Kommutativgesetz Addition $\mathbb{N}$}\\
& = \overline{((ac+bd)+(ae+bf)),((ad+bc)+(af+be))}   \tag{Assoziativgesetz Addition $\mathbb{N}$}\\
& = \overline{((ac+bd),(ad+bc))} + \overline{((ae+bf),(af+be))}   \tag{Addition $\mathbb{Z}$}\\
& = \overline{(a,b)} \cdot \overline{(c,d)} + \overline{(a,b)} \cdot \overline{(e,f)} \tag{Multiplikation $\mathbb{Z}$}  ~~_\Box\\
\end{align*}
\section*{Aufgabe B3}

$F_n$ ist rekursiv definiert:
\begin{itemize}
\item $F_0 = 1 ; F_1 = 1$
\item für $n \in \mathbb{N}, n > 1$: $F_n = F_{n+1} + F_{n+2}$
\end{itemize}
Behauptung:

Für alle $ n \in \mathbb{N}$ gilt:

$ 2F_0 + \sum \limits_{k=1}^n F_k = F_{n+2}$ \\

Beweis durch vollständige Induktion über n.

\textbf{Induktionsanfang:}

$ n = 0$ \\
$ 2F_0 + \sum \limits_{k=1}^0 F_k = F_{0+2}$ ($ \sum \limits_{k=1}^0 F_k = 0$, da kein Summand vorhanden ist) \\
$2F_0 = F_2$ \\
$2\cdot1 = F_1 + F_0$\\
$2 = 1 + 1$ \\
$2 = 2$ $\surd$ \\

\textbf{Induktionsvoraussetzung:} \\
Für beliebiges aber festes n gilt:
$ 2F_0 + \sum \limits_{k=1}^n F_k = F_{n+2}$ \\

\textbf{Induktionsschritt:} \\
z.z.: $ 2F_0 + \sum \limits_{k=1}^{n+1} F_k = F_{(n+1)+2}$ \\

$ 2F_0 + \sum \limits_{k=1}^{n+1} F_k$ \\
$= 2F_0 + F_{n+1} + \sum \limits_{k=1}^n F_k$ \\
$= F_{n+1} + 2F_0 + \sum \limits_{k=1}^n F_k$ \\
$= F_{n+1} + F_{n+2}$   (nach \textbf{Induktionsvoraussetzung}) \\
$= F_{n+3}$   (nach Definition Fibonacci-Zahlen) \\
$= F_{(n+1)+2}$    $~~~~ _\Box$

\end{document}
